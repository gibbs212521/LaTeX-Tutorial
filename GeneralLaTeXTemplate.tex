%		For public use only, fair use laws still apply
%		A tutorial template created by Chad Gibbons




%	I recommend for generic help the following website
%	http://www.emerson.emory.edu/services/latex/latex_toc.html
%	There are many good websites and forums for LaTeX
%
%	If you have a problem, google it. (ex. "How do I start my page numbers at a different number")



%%%%%%%%%%%%%%%%%%%%%%%%%%%%%%%%%%%%%%%%%%%%%%%%%%%%%%
%					 													  %
%					 													  %
%					  ALWAYS BEGIN THE PROGRAM WITH								  %
%					 													  %
%						\documentclass{class_type}								  %
%																		  %
%		Different class types have different commands associated with them, similar to header files			  %
%					 													  %
%					 													  %
%%%%%%%%%%%%%%%%%%%%%%%%%%%%%%%%%%%%%%%%%%%%%%%%%%%%%%

\documentclass{article}

%%%%%%%%%%%%%%%%%%%%%%%%%%%%%%%%%%%%%%%%%%%%%%%%%%%%%%
%					 													  %
%					 													  %
%				It is recommended to only use packages as you need them 						  %
%					 													  %
%	Different types of reports will consistently use specific packages; keep these in their respective templates		  %
%					 													  %
%					 													  %
%%%%%%%%%%%%%%%%%%%%%%%%%%%%%%%%%%%%%%%%%%%%%%%%%%%%%%


\usepackage{babel}				%	Expands text mode
\usepackage{csquotes}				%	Permits \enquote{quote}
\usepackage{graphicx}				%	Permits \includegraphics[]{image}
\usepackage{caption}				%	Permits the void caption \caption*{}


%%%%%%%%%%%%%%%%%%%%%%%%%%%%%%%%%%%%%%%%%%%%%%%%%%%%%%%			PREAMBLE


\title{Introduction to \LaTeX{\textquotedblleft Lay Tech"}}		%note left quote command vs " = \textquotedblright
\author{Author's Name}
%\date{12/29/2018}					%	Set permanent date or use date section as additional text real estate on the title page.
%\date{Sic et Non \\ Quid Faciendus Est}



%%%%%%%%%%%%%%%%%%%%%%%%%%%%%%%%%%%%%%%%%%%%%%%%%%%%%%%			DOCUMENT





%%%%%%%%%%%%%%%%%%%%%%%%%%%%%%%%%%%%%%%%%%%%%%%%%%%%%%
%					 													  %
%					 													  %
%					  ALWAYS BEGIN THE DOCUMENT WITH								  %
%					 													  %
%							\begin{document}									  %
%					 													  %
%					 													  %
%%%%%%%%%%%%%%%%%%%%%%%%%%%%%%%%%%%%%%%%%%%%%%%%%%%%%%


\begin{document}

\maketitle

	%%%%%%%%%%%%%%%%%%%%%%%%%%%%%%%%%%%%%%%%%%
	%														 %
	%														 %
	%		\hfill fills in the remaining spaces of the given line with spaces			 %
	%														 %
	%					\\ ends line           							 %
	%														 %
	%	Determine the number of "\hfill \\ " iterations in accord with the number of lines		 %
	%				   within you abstract.							 %
	%														 %
	%														 %
	%%%%%%%%%%%%%%%%%%%%%%%%%%%%%%%%%%%%%%%%%%

\hfill 
\hfill \\
\hfill \\
\hfill \\
\hfill \\
\hfill \\
\hfill \\
\hfill \\
\hfill \\
\hfill \\
\hfill \\
\hfill \\
\hfill \\
\hfill \\
\hfill \\
\hfill \\
\hfill \\
\hfill \\
\hfill \\
\hfill \\
\hfill \\
\hfill


	%%%%%%%%%%%%%%%%%%%%%%%%%%%%%%%%%%%%%%%%%%
	%														 %
	%														 %
	%		The primary purpose of an abstract is to give sufficient information		 %
	%					on your report to inform 						 %
	%		researchers, students, grant providers, corporate executives, professors,  	 %
	%				professionals, or the audience of your intention			 %
	%		whether or not either reading or purchasing your report is necessary.		 %
	%														 %
	%														 %
	%%%%%%%%%%%%%%%%%%%%%%%%%%%%%%%%%%%%%%%%%%

\begin{abstract}
\noindent The abstract text goes here.\\
\hfill \\
NOTE: Attention to comments and visual errors.\\
\qquad \quad Some errors are purposely inserted with the method of correction \\
\indent \qquad following it immediately.
\end{abstract}

\pagenumbering{gobble}

\pagebreak


\tableofcontents



	%%%%%%%%%%%%%%%%%%%%%%%%%%%%%%%%%%%%%%%%%%
	%														 %
	%														 %
	%		\pagenumbering{gobble}	"gobbles" or removes the pagination of the page  	 %
	%					and the following pages.						 %
	%														 %
	%		\pagenumbering{num_style}	sets the pagination of the page and 	  	 %
	%						the following pages						 %
	%														 %
	%						{num_style} List:						 %
	%														 %
	%				{roman} -> lowercase roman numerals           				 %
	%				{Roman} -> uppercase roman numerals           				 %
	%				{alph} -> lowercase alphabetical numerals           			 %
	%				{Alph} -> uppercase alphabetical numerals           			 %
	%				{arabic} -> standard decimal numerals (0-9)           			 %
	%														 %
	%			Altering \pagenumbering{} will restart the pagination at 1.			 %
	%			     See other tutorial for greater control of paginations.			 %
	%														 %
	%														 %
	%%%%%%%%%%%%%%%%%%%%%%%%%%%%%%%%%%%%%%%%%%


\pagenumbering{roman}

\pagebreak




	%%%%%%%%%%%%%%%%%%%%%%%%%%%%%%%%%%%%%%%%%%
	%														 %
	%														 %
	%	The primary purpose of an introduction is to provide a fundamental  review		 %
	%				of topics in your report to prepare 					 %
	%		researchers, students, grant providers, corporate executives, professors,  	 %
	%				professionals, or the audience of your intention			 %
	%	with sufficient explanation that someone of appropriate technical knowledge		 %
	%			would be able to comprehend the method and conclusion 			 %
	%														 %
	%														 %
	%%%%%%%%%%%%%%%%%%%%%%%%%%%%%%%%%%%%%%%%%%


\section{Introduction}
Here is the text of your introduction.

\pagenumbering{arabic}

						

\begin{equation}
    \label{simple_equation}
    \alpha = \sqrt{ \beta }
\end{equation}

\hfill \\


$$ \alpha \cdot \beta = \alpha^{1.5}_{subscript} $$		% NOTE: No equation number nor tie to any list

%\text{test = } 
test = $\int^\infty_{-\infty} $u(t)$ \cdot$ (- u(t-3))$ $dx
\begin{center}
The answer to what test is equal to is: $\int^3_0$ 1 dx\\
test = 3
\end{center}

\subsection{Subsection Heading Here}
Write your subsection text here.


\subsection*{Subsection Heading Here \& NOT in TOC}

TOC stands for Table of Contents. \\

\noindent $\backslash$\& permits the writting of \& in text.\\
\indent Notice also that there is no iterative numeration for any section, subsection, table, or figure that has the asterisk (*) infront of the call-command and the bracketed definition.


% save an image in the same local file as the LaTeX file with the desired name (e.g. myfigure.jpg for below)

%\begin{figure}[h!]
   % \centering
   % \includegraphics[width=3.0in]{myfigure}
   % \caption{Simulation Results}
   % \label{simulationfigure}
%\end{figure}

%\begin{figure}[h!]
%	\centering
%	\includegraphics[size=0.75]{myfigure}
%	\caption{size alteration example}
%	\label{Label Location}
%\end{figure}

\section{Text vs Math Formating}

The standard input mode is text input. \\
\hfill \\

The alternative input mode is math or symbol input. \\
\hfill \\

\noindent You may use inline equations by inserting the dollar sign in the same manner as quoting with quotation marks (`` ") \$ $\backslash$frac\{ numerator\}\{denominator\} \$ to create:  \\


	%%%%%%%%%%%%%%%%%%%%%%%%%%%%%%%%%%%%%%%%%%
	%														 %
	%														 %
	%					Special Note on Quotation Marks					 %
	%														 %
	%			   Standard Quotation Mark (") --> right quotation mark			 %
	%			Double Reverse Accent Mark (``) --> left quotation mark	 		 %
	%			left quotation mark = \textquotedblleft vs \textquotedblright 		 %
	%														 %
	%														 %
	%%%%%%%%%%%%%%%%%%%%%%%%%%%%%%%%%%%%%%%%%%


\indent \qquad \quad  $\frac{numerator}{denominator}$		%\qquad is a quintuplet space vs quad which is a quatruple space insert (5 vs 4 spaces :: \qquad vs \quad)


\pagebreak

\section{Table Tutorial}


	%%%%%%%%%%%%%%%%%%%%%%%%%%%%%%%%%%%%%%%%%%
	%														 %
	%														 %
	%					Special Note on Tables						 %
	%														 %
	%				   l = left, c = center, r = right alignment				 %
	%			   | in tabular sets borders between alignment sets		 		 %
	%	      \hline places a horizontal line across width of table above current the line  		 %
	%			   \\ (the endline command) begins next line of table.		 		 %
	%														 %
	%														 %
	%%%%%%%%%%%%%%%%%%%%%%%%%%%%%%%%%%%%%%%%%%

\begin{table}[h!]				% [h!] gives the table a set place in the document
\caption{Ecce hoc titulum pro tabulo super illo!}
\centering
\begin{tabular}{ l | c | c | r | }
left & center & center & right \\
no left border & no top border & bottom border & right border \\ 
\hline 
no left & top & bottom & right \\
\hline
no left & top & no bottom & right \\


\end{tabular}
\end{table}


\begin{table}[h!]
\begin{tabular}{| r | c | c | l |}
\hline
 left & top & bottom & right \\
\hline
right alignment & center alignment & center alignment & left alignment \\
\hline
&&&\\
\hline
\textbf{Species} & number in left group & number in right group & total number \\
\hline
dogs & 3 & 4 & 7 \\
\hline
cats & 54 & 36 & 90 \\
\hline

\end{tabular}
\caption*{Ecce hoc titulum pro ipsum cum nulla numeris}
\end{table}


\begin{table}[h!]				% [h!] gives the table a set place in the document
\centering
\begin{tabular}{ l | c | c | r | }
left & center & center & right \\
no left border & no top border & bottom border & right border \\ 
\hline 
no left & top & bottom & right \\
\hline
no left & top & no bottom & right \\
\end{tabular}
\caption{Ecce hoc titulum pro tabulo sub illo!}
\end{table}

\pagebreak

\section{Method}

\section{Results}

\section{Discussion}

\section{Conclusion}
Write your conclusion here.

\pagebreak

\section{Internal Figures \& Tables Reference Lists}

\listoftables

\listoffigures

\section{References}

\section{Table of Appendices}

\section*{Appendix A}

\section*{Appendix B}

\end{document}




%%%%%%%%%%%%%%%%%%%%%%%%%%%%%%%%%%%%%%%%%%%%%%%%%%%%%%
%					 													  %
%					 													  %
%					  ALWAYS END THE DOCUMENT WITH								  %
%					 													  %
%							\end{document}									  %
%					 													  %
%					 													  %
%%%%%%%%%%%%%%%%%%%%%%%%%%%%%%%%%%%%%%%%%%%%%%%%%%%%%%